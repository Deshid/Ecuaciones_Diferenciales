\documentclass[a4paper,12pt,numbers=noenddot]{scrreprt}
\setlength{\headheight}{61.24997pt} % Definition der KOMA-Textklasse
\usepackage{amsmath}
\usepackage{tikz} % Add this line to include the tikz package
\usepackage{fancyhdr} 
\usepackage{graphicx}
\usepackage{ragged2e} % Add this line to include the ragged2e package
\usepackage{multicol} % Add this line to include the multicol package
\usepackage{amsfonts}

%%%%%%%%%%%%%%%%%%%%%%
% Set up fancy header/footer
\pagestyle{fancy}
\fancyhead[LO,L]{Deshi}
\fancyhead[CO,C]{Ecuaciones diferenciales - Fórmulas}
\fancyhead[RO,R]{\today}
\fancyfoot[LO,L]{}
\fancyfoot[CO,C]{\thepage}
\fancyfoot[RO,R]{}
\renewcommand{\headrulewidth}{0.4pt}
\renewcommand{\footrulewidth}{0.4pt}
%%%%%%%%%%%%%%%%%%%%%%

\begin{document}

% Add the logo in the header
\lhead{\includegraphics[width=2cm]{LogoCat.png}} % Replace "logo.png" with the filename of your logo image

\title{Formulas}

% Contenido del documento
\section*{Integrales inmediatas}
    \begin{multicols}{2} % This starts a three-column layout
        \begin{equation*}
            \int k \, dx = kx + C
        \end{equation*}
        \begin{equation*}
            \int x^n \, dx = \frac{x^{n+1}}{n+1} + C
        \end{equation*}
        \begin{equation*}
            \int e^x \, dx = e^x + C
        \end{equation*}
        \begin{equation*}
            \int e^{kx} \, dx = \frac{e^{kx}}{k} + C
        \end{equation*}
        \begin{equation*}
            \int \frac{1}{x} \, dx = \ln |x| + C
        \end{equation*}
        \begin{equation*}
            \int a^x \, dx = \frac{a^x}{\ln a} + C
        \end{equation*}
        \begin{equation*}   
            \int \sin x \, dx = -\cos x + C
        \end{equation*}
        \begin{equation*}
            \int \cos x \, dx = \sin x + C
        \end{equation*}
        \begin{equation*}
            \int \cos 2x \, dx = \frac{\sin 2x}{2}  + C
        \end{equation*}
        \begin{equation*}
            \int \sin 2x \, dx = -\frac{\cos 2x}{2} + C 
        \end{equation*}
        \begin{equation*}
            \int \frac{1}{\cos^2 x} \, dx = \tan x + C
        \end{equation*}
        \begin{equation*}
            \int \frac{1}{\sin^2 x} \, dx = -\cot x + C
        \end{equation*}
        \begin{equation*}
            \int \sec x \, dx = \ln |\sec x + \tan x| + C
        \end{equation*}
        \begin{equation*}
            \int \sec^2 x \, dx = \tan x + C
        \end{equation*}
        \begin{equation*}
            \int \sec x \, \tan x \, dx = \sec x + C
        \end{equation*}
        \begin{equation*}
            \int \ln x \, dx = x \ln x - x + C
        \end{equation*}
        \begin{align*}
        \int \tan x \, dx = -\ln |\cos x| + C\\=\ln |\sec x| + C
        \end{align*}
        \begin{equation*}
        \int \tan^2 x \, dx = \tan x - x + C   
        \end{equation*}
        \begin{equation*}
        \int x \sin x \, dx = \sin x - x \cos x + C
        \end{equation*}
        \begin{equation*}
        \int x \cos x \, dx = \cos x + x \sin x + C
        \end{equation*}

    \end{multicols}

    \subsection*{Integral por partes}
        \begin{equation*}
            \int u \, dv = uv - \int v \, du
        \end{equation*}

\section*{Derivadas}
        \begin{center} 
    \begin{multicols}{4}
        \begin{align*}       
        \textbf{Regla de la cadena:}\\ (f(g))' = f'g \cdot g'
        \end{align*}
        \begin{align*}       
            \textbf{Regla del producto}\\ (f \cdot g)' = f'g + fg'
        \end{align*}
        \begin{equation*}
            (k)' = 0
        \end{equation*}
        \begin{equation*}
            (x)' = 1
        \end{equation*}
        \begin{equation*}
            (n x)' = x
        \end{equation*}
        \begin{equation*}
            (x^n)' = nx^{n-1}
        \end{equation*}
        \begin{equation*}
            (e^x)' = e^x
        \end{equation*}
        \begin{equation*}
            (e^{kx})' = ke^{kx}
        \end{equation*}
        \begin{equation*}
            (\ln |x|)' = \frac{1}{x}
        \end{equation*}
        \begin{equation*}
            (a^x)' = a^x \ln a
        \end{equation*}
        \begin{equation*}
            (\sin x)' = \cos x
        \end{equation*}
        \begin{equation*}
            (\cos x)' = -\sin x
        \end{equation*}
        \begin{equation*}
            (\tan x)' = \sec^2 x
        \end{equation*}
        \begin{equation*}
            (\cot x)' = -\csc^2 x
        \end{equation*}
        \begin{equation*}
            (\sec x)' = \sec x \tan x
        \end{equation*}
        \begin{equation*}
            (\csc x)' = -\csc x \cot x
        \end{equation*}
        \begin{equation*}
            (\frac{1}{x})' = -\frac{1}{x^2}
        \end{equation*} 
        \begin{align*}
            (x^{-2})' = -2x^{-3}\\
            = -\frac{2}{x^3}
        \end{align*}
        \begin{equation*}
            (\sqrt{x})' = \frac{1}{2\sqrt{x}}
        \end{equation*}
    \end{multicols}
\end{center}
    \newpage

    \section*{Identidades trigonométricas}
    \begin{multicols}{3}
        \begin{equation*}
            \sin^2 x + \cos^2 x = 1
        \end{equation*}
        \begin{equation*}
            \tan x = \frac{\sin x}{\cos x}
        \end{equation*}
        \begin{equation*}
            \cot x = \frac{\cos x}{\sin x}
        \end{equation*}
        \begin{equation*}
            \sec x = \frac{1}{\cos x}
        \end{equation*}
        \begin{equation*}
            \csc x = \frac{1}{\sin x}
        \end{equation*}
        \begin{equation*}
            \sin 2x = 2 \sin x \cos x
        \end{equation*}
        \begin{equation*}
            \cos 2x = \cos^2 x - \sin^2 x
        \end{equation*}
        \begin{equation*}
            \cos 2x = 2 \cos^2 x - 1
        \end{equation*}
        \begin{equation*}
            \cos 2x = 1 - 2 \sin^2 x
        \end{equation*}
        \begin{equation*}
            \sin^2 x = \frac{1 - \cos 2x}{2}
        \end{equation*}
        \begin{equation*}
            \cos^2 x = \frac{1 + \cos 2x}{2}
        \end{equation*}
        \begin{equation*}
        \tan^2 x +1 = \sec^2 x
        \end{equation*}
        \begin{equation*}
            \cot^2 x + 1 = \csc^2 x
        \end{equation*}
    \end{multicols}

    \section*{Propiedades}
    \begin{multicols}{3}
        \begin{equation*}
            \scalebox{1.4}{$\frac{u^{\frac{3}{5}}}{\frac{3}{5}} = \frac{5}{3}u^{\frac{3}{5}}$}
        \end{equation*}
        \begin{equation*}
            \scalebox{1.3}{$\frac{1}{u^{2/5}} = u^{-2/5}$}
        \end{equation*}
        \begin{equation*}
            \scalebox{1.2}{$^n \sqrt{u^m} = u^{m/n}$}
        \end{equation*}    
        \begin{equation*}
            \scalebox{1.2}{$\ln(e^x) = x$}
        \end{equation*}
        \begin{equation*}
            \scalebox{1.2}{$e^{\ln x} = x$}	
        \end{equation*}
        \end{multicols}
\section*{Teoremas de ecuaciones diferenciales}
\subsection*{Teorema 1:}
\begin{align*}
    &\textbf{Resolución de una E.D. de variables separables:}\\
    &\text{Dada la E.D. de variables separables M(x)\,dx + N(y)\,dy = 0, la solución es:}
\end{align*}
\begin{center}
    \(\int M(x)\,dx + \int N(y)\,dy = C\)
\end{center}
\subsection*{Teorema 2:}
\begin{align*}
    &\textbf{Resolución de una E.D. homogénea:}\\
    &\text{Dada la E.D. homogénea M(x,y)\,dx + N(x,y)\,dy = 0, tomamos el cambio de variable y=ux,}\\ 
    &\text{transformandola en una E.D. de variable separable.}
\end{align*}
\begin{center}
    \(y=ux\)\\
    \(dy = u\,dx + x\,du\)
\end{center}

\subsection*{Teorema 3:}
\begin{align*}
    &\textbf{Resolución de una E.D. exacta:}\\
    &\text{Dada la E.D. exacta M(x,y)\,dx + N(x,y)\,dy = 0, existe una función F(x,y) tal que:}\\
    &\frac{\partial F}{\partial x} = M(x,y) \quad \text{y} \quad \frac{\partial F}{\partial y} = N(x,y)
\end{align*}

\subsection*{Teorema 4:}
\begin{align*}
    &\textbf{Factores integrantes:}\\
    &\text{Si la E.D. de la forma M(x,y)\,dx + N(x,y)\,dy = 0 no es exacta, entonces:}\\
    &\textbf{A:} \quad \text{Si} \quad \frac{\frac{\partial M}{\partial y} - \frac{\partial N}{\partial x}}{N(x,y)} \quad \text{sólo depende de "x", entonces el factor integrante es:}\\
    &\mu(x) = e^{\int \frac{\frac{\partial M}{\partial y} - \frac{\partial N}{\partial x}}{N(x,y)} \,dx}\\
    &\textbf{B:} \quad \text{Si} \quad \frac{\frac{\partial N}{\partial x} - \frac{\partial M}{\partial y}}{M(x,y)} \quad \text{sólo depende de "y", entonces el factor integrante es:}\\
    &\mu(y) = e^{\int \frac{\frac{\partial N}{\partial x} - \frac{\partial M}{\partial y}}{M(x,y)} \,dy}
\end{align*}
\subsection*{Teorema 5:}
\begin{align*}
    &\textbf{Resolución de una E.D. lineal de primer orden:}\\
    &\text{Dada la E.D. lineal de primer orden:} \quad \frac{dy}{dx} + p(x)y = q(x), \quad \text{La solución es:}\\
    &y = e^{-\int p(x)\,dx} \left( C + \int q(x) e^{\int p(x)\,dx} \,dx \right)
\end{align*}
\subsection*{Teorema 6:}
\begin{align*}
    &\textbf{Resolución de una E.D. Bernoulli}\\
    &\text{Dada la E.D. de Bernoulli:} \quad \frac{dy}{dx} + p(x)y = q(x)y^n \quad (n \neq 1, \quad n\neq 0)\\
    &\text{Tomamos el cambio de variable:} \quad u = y^{1-n}\\
    &\text{Esto transforma la E.D. de Bernoulli en una E.D. lineal de primer orden.}
\end{align*}
\subsection*{Teorema 7:}
\begin{align*}
    &\textbf{Resolución de una E.D. Riccati}\\
    &\text{Dada la E.D. de Riccati:} \quad \frac{dy}{dx} + a_2(x)y^2 + a_1(x)y + a_0(x)=0 \quad (a_0(x) \neq 0 \quad \wedge \quad a_2(x)\neq 0)\\
    &\text{siendo} \quad y_{particular}(x) \quad \text{una solución, entonces:}\\
    &\textbf{A:} \quad \text{Si tomamos el cambio de variable:} \quad y = y_{p}(x) + u\\
    &\text{La E.D. de Riccati se transforma en una E.D. de Bernoulli.}\\
    &\textbf{B:} \quad \text{Si tomamos el cambio de variable} \quad y = y_p(x) + \frac{1}{u}\\
    &\text{La E.D. de Riccati se transforma en una E.D. lineal de primer orden.}
\end{align*}
        \end{document}
