\documentclass[a4paper,12pt,numbers=noenddot]{scrreprt}
\setlength{\headheight}{61.24997pt} % Definition der KOMA-Textklasse
\usepackage{amsmath}
\usepackage{tikz} % Add this line to include the tikz package
\usepackage{fancyhdr} 
\usepackage{graphicx}
\usepackage{ragged2e} % Add this line to include the ragged2e package
\usepackage{multicol} % Add this line to include the multicol package
\usepackage{amsfonts}

%%%%%%%%%%%%%%%%%%%%%%
% Set up fancy header/footer
\pagestyle{fancy}
\fancyhead[LO,L]{Deshi}
\fancyhead[CO,C]{Ecuaciones diferenciales - Apunte}
\fancyhead[RO,R]{\today}
\fancyfoot[LO,L]{}
\fancyfoot[CO,C]{\thepage}
\fancyfoot[RO,R]{}
\renewcommand{\headrulewidth}{0.4pt}
\renewcommand{\footrulewidth}{0.4pt}
%%%%%%%%%%%%%%%%%%%%%%

\begin{document}

% Add the logo in the header
\lhead{\includegraphics[width=2cm]{LogoCat.png}} % Replace "logo.png" with the filename of your logo image

\title{Formulas}

% Contenido del documento
\section*{Fórmulas básicas a recordar}
\subsection*{Integrales inmediatas}
    \begin{multicols}{2} % This starts a three-column layout
        \begin{equation*}
            \int k \, dx = kx + C
        \end{equation*}
        \begin{equation*}
            \int x^n \, dx = \frac{x^{n+1}}{n+1} + C
        \end{equation*}
        \begin{equation*}
            \int e^x \, dx = e^x + C
        \end{equation*}
        \begin{equation*}
            \int \frac{1}{x} \, dx = \ln |x| + C
        \end{equation*}
        \begin{equation*}
            \int a^x \, dx = \frac{a^x}{\ln a} + C
        \end{equation*}
        \begin{equation*}   
            \int \sin x \, dx = -\cos x + C
        \end{equation*}
        \begin{equation*}
            \int \cos x \, dx = \sin x + C
        \end{equation*}
        \begin{equation*}
            \int \cos 2x \, dx = \frac{\sin 2x}{2}  + C
        \end{equation*}
        \begin{equation*}
            \int \sin 2x \, dx = -\frac{\cos 2x}{2} + C 
        \end{equation*}
        \begin{equation*}
            \int \frac{1}{\cos^2 x} \, dx = \tan x + C
        \end{equation*}
        \begin{equation*}
            \int \frac{1}{\sin^2 x} \, dx = -\cot x + C
        \end{equation*}
        \begin{equation*}
            \int \sec x \, dx = \ln |\sec x + \tan x| + C
        \end{equation*}
        \begin{equation*}
            \int \sec^2 x \, dx = \tan x + C
        \end{equation*}
        \begin{equation*}
            \int \sec x \, \tan x \, dx = \sec x + C
        \end{equation*}
        \begin{equation*}
            \int \ln x \, dx = x \ln x - x + C
        \end{equation*}
        \begin{align*}
        \int \tan x \, dx = -\ln |\cos x| + C\\=\ln |\sec x| + C
        \end{align*}
        \begin{equation*}
        \int \tan^2 x \, dx = \tan x - x + C   
        \end{equation*}
        \begin{equation*}
        \int x \sin x \, dx = \sin x - x \cos x + C
        \end{equation*}
        \begin{equation*}
        \int x \cos x \, dx = \cos x + x \sin x + C
        \end{equation*}

    \end{multicols}

    \subsection*{Integral por partes}
        \begin{equation*}
            \int u \, dv = uv - \int v \, du
        \end{equation*}

\subsection*{Derivadas}
        \begin{center} 
    \begin{multicols}{4}
        \begin{align*}       
        \textbf{Regla de la cadena:}\\ (f(g))' = f'g \cdot g'
        \end{align*}
        \begin{align*}       
            \textbf{Regla del producto}\\ (f \cdot g)' = f'g + fg'
        \end{align*}
        \begin{equation*}
            (k)' = 0
        \end{equation*}
        \begin{equation*}
            (x)' = 1
        \end{equation*}
        \begin{equation*}
            (x^2)' = 2x
        \end{equation*}
        \begin{equation*}
            (x^n)' = nx^{n-1}
        \end{equation*}
        \begin{equation*}
            (e^x)' = e^x
        \end{equation*}
        \begin{equation*}
            (\ln |x|)' = \frac{1}{x}
        \end{equation*}
        \begin{equation*}
            (a^x)' = a^x \ln a
        \end{equation*}
        \begin{equation*}
            (\sin x)' = \cos x
        \end{equation*}
        \begin{equation*}
            (\cos x)' = -\sin x
        \end{equation*}
        \begin{equation*}
            (\tan x)' = \sec^2 x
        \end{equation*}
        \begin{equation*}
            (\cot x)' = -\csc^2 x
        \end{equation*}
        \begin{equation*}
            (\sec x)' = \sec x \tan x
        \end{equation*}
        \begin{equation*}
            (\csc x)' = -\csc x \cot x
        \end{equation*}
        \begin{equation*}
            (\frac{1}{x})' = -\frac{1}{x^2}
        \end{equation*} 
    \end{multicols}
\end{center}
    \newpage

    \subsection*{Identidades trigonométricas}
    \begin{multicols}{3}
        \begin{equation*}
            \sin^2 x + \cos^2 x = 1
        \end{equation*}
        \begin{equation*}
            \tan x = \frac{\sin x}{\cos x}
        \end{equation*}
        \begin{equation*}
            \cot x = \frac{\cos x}{\sin x}
        \end{equation*}
        \begin{equation*}
            \sec x = \frac{1}{\cos x}
        \end{equation*}
        \begin{equation*}
            \csc x = \frac{1}{\sin x}
        \end{equation*}
        \begin{equation*}
            \sin 2x = 2 \sin x \cos x
        \end{equation*}
        \begin{equation*}
            \cos 2x = \cos^2 x - \sin^2 x
        \end{equation*}
        \begin{equation*}
            \cos 2x = 2 \cos^2 x - 1
        \end{equation*}
        \begin{equation*}
            \cos 2x = 1 - 2 \sin^2 x
        \end{equation*}
        \begin{equation*}
            \sin^2 x = \frac{1 - \cos 2x}{2}
        \end{equation*}
        \begin{equation*}
            \cos^2 x = \frac{1 + \cos 2x}{2}
        \end{equation*}
        \begin{equation*}
        \tan^2 x +1 = \sec^2 x
        \end{equation*}
        \begin{equation*}
            \cot^2 x + 1 = \csc^2 x
        \end{equation*}
    \end{multicols}

    \subsection*{Propiedades}
        \begin{equation*}
            \scalebox{1.5}{$\frac{u^{\frac{3}{5}}}{\frac{3}{5}} = \frac{5}{3}u^{\frac{3}{5}}$}
        \end{equation*}
        \begin{equation*}
            \scalebox{1.5}{$\frac{1}{u^{2/5}} = u^{-2/5}$}
        \end{equation*}
        \end{document}
